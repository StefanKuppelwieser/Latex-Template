%%%%%%%%%%%%%%%%%%%%%%%%%%%%%%%%%%%%%%%%%%%%%%%%%%%%%%%%%%%%%
%%
%% Autoren		        Sebastian Büchler
%%				OpenSource@sbuechler.de
%%				https://github.com/sebikolon
%%				
%%				Stefan Kuppelwieser
%%				OpenSource@kuppelwieser.net
%%				https://github.com/StefanKuppelwieser
%%
%%%%%%%%%%%%%%%%%%%%%%%%%%%%%%%%%%%%%%%%%%%%%%%%%%%%%%%%%%%%%

%%%%%%%%%%%%%%%%%%%%%%%%%%%%%%%%%%%%%%%%%%%%%%%%%%%%%%%%%%%%%
%%
%%	Die Verwendung der Elemente erhebt keinen Anspruch auf Vollständigkeit.
%%
%% 	Das Template steht unter der MIT-Lizenz.
%%
%%%%%%%%%%%%%%%%%%%%%%%%%%%%%%%%%%%%%%%%%%%%%%%%%%%%%%%%%%%%%

%Dokumentklasse
\documentclass[a4paper,11pt]{scrreprt}
% Raender des Dokuemnts einstellen
\usepackage[left= 3.0 cm ,right = 3.0 cm, bottom = 3.0 cm, top = 3.0 cm]{geometry}
\usepackage[onehalfspacing]{setspace}
% ============= Packages =============
% Dokumentinformationen
\usepackage[
	pdftitle={Latex-Template},
	pdfsubject={Latex-Template},
	pdfauthor={Stefan Kuppelwieser und Sebastian Büchler},
	pdfkeywords={},	
	%Links nicht einrahmen
	hidelinks,
	colorlinks=true, % set colors for links and URLs
	urlcolor=blue, % set colors for links and URLs
	linkcolor=blue % set colors for links and URLs
]{hyperref}

% ============= Standard Packages =============
\usepackage{svg}
\usepackage{calc}
\usepackage{booktabs}
\usepackage[utf8]{inputenc}
\usepackage[ngerman]{babel}			 	% Für BibLatex
\usepackage[babel, german=quotes]{csquotes}		% Für BibLatex
\usepackage[T1]{fontenc}
\usepackage{graphicx}
\usepackage{subcaption}
\graphicspath{{img/}}
\usepackage{fancyhdr}
\usepackage{array}
\usepackage{listings}
\usepackage{multirow} 				 	% Tabellen-Zellen ueber mehrere Zeilen
\usepackage{multicol} 				 	% mehre Spalten auf eine Seite
\usepackage{tabularx}                			% Fuer Tabellen mit vorgegeben Groessen
\usepackage{lmodern}
\usepackage{color}
\usepackage{placeins}
%\usepackage{subfig}
\usepackage{xr}
% zusätzliche Schriftzeichen der American Mathematical Society
\usepackage{amsfonts}
\usepackage{amsmath}
\usepackage{pdflscape}              			% Einzelne-Seite-im-Querformat
\usepackage{pdfpages} 
\usepackage{float}
\restylefloat{figure}
\usepackage{threeparttable}        			% Für Fußnoten ind er Tabelle
\usepackage{epstopdf} 					% enable eps graphics
\usepackage{svg}					% Um Vektorbilder hinzufügen zu können
\usepackage{chngcntr}					% Für fortlaufende Fußnoten im gesammten Dokument

% ============= Erlaubt überall Umlaute =============
\lstset{literate=%
    {Ö}{{\"O}}1
    {Ä}{{\"A}}1
    {Ü}{{\"U}}1
    {ß}{{\ss}}1
    {ü}{{\"u}}1
    {ä}{{\"a}}1
    {ö}{{\"o}}1
    {~}{{\textasciitilde}}1
}

% ============= Fortlaufende Fußnoten =============
\counterwithout{footnote}{chapter}

% ============= Formatierung fuer Quellcode-Listings =============
\lstset{
   numbers=none,
   tabsize=3,
   breaklines=true,
   basicstyle=\small\ttfamily,
   framerule=0pt,
   backgroundcolor=\color{gray!25},
   columns=fullflexible
}

% ============= Literaturverzeichnis =============
% WICHTIG: Hier wird nicht BibTeX sondern BibLateX verwendet!!
% Deshalb nicht mit bibtex uebersetzen, sondern mit biber
% Das kann man in jedem Tool wie TexMaker oder TexShop als Option einstellen
\usepackage[backend=biber, style=alphabetic]{biblatex}
%\usepackage[backend=biber, style=numeric-comp, sorting=none]{biblatex}

% Hier werden die Referenzen in einer separaten Datei gespeichert
\addbibresource{Literatur.bib}

% Maximale Tiefe des Inhaltsverzeichnisses
\setcounter{tocdepth}{1}

\ExecuteBibliographyOptions{%
	sorting=nty,		% Sortiert alphabetisch nach Name, Titel und Jahr
	isbn=false,
	url=false,
	doi=false,
	eprint=false,
	firstinits=true,
    maxbibnames=99,   	% Alle Autoren (kein et al.)
    backref=false,    	% keine Ruueckverweise auf Zitatseiten    
   }
   
\renewcommand*{\labelalphaothers}{} % alpha label ohne +

% ============= Bildunterschrift =============
\setcapindent{0em} % kein Einruecken der Caption von Figures und Tabellen
\setcapwidth{0.9\textwidth}
\setlength{\abovecaptionskip}{0.2cm} % Abstand der zwischen Bild- und Bildunterschrift

% Nicht einruecken nach Absatz
\setlength{\parindent}{0pt}

% ============= Kopf- und Fußzeile =============
\pagestyle{fancy}
\fancyhf{}
\fancyhead[LE,RO]{\slshape \leftmark}
%\fancyhead[RE,LO]{Guides and tutorials}
%\fancyfoot[CE,CO]{\leftmark}
\fancyfoot[LE,RO]{\thepage}
%%
\renewcommand{\headrulewidth}{0.4pt}
\renewcommand{\footrulewidth}{0pt}

% ============= Tabelle =============
\renewcommand{\arraystretch}{1.4} %Zuständig für die Zeilenhöhe

% ============= Package Einstellungen & Sonstiges ============= 
%Besondere Trennungen
\hyphenation{De-zi-mal-tren-nung}

% ============= Erzwingen eines Zeilenumbruchs innerhalb einer URL =============
\renewcommand{\UrlBreaks}{\do\/\do\a\do\b\do\c\do\d\do\e\do\f\do\g\do\h\do\i\do\j\do\k\do\l\do\m\do\n\do\o\do\p\do\q\do\r\do\s\do\t\do\u\do\v\do\w\do\x\do\y\do\z\do\A\do\B\do\C\do\D\do\E\do\F\do\G\do\H\do\I\do\J\do\K\do\L\do\M\do\N\do\O\do\P\do\Q\do\R\do\S\do\T\do\U\do\V\do\W\do\X\do\Y\do\Z}
